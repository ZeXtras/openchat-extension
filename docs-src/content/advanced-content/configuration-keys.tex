\section{Configuration Keys}
\label{==sect:confkeys==}

OpenChat extension is easily configurable through the Zimbra CLI, all of the congurations are store in LDAP.\\

To edit an account configuration, for example run these commands:
\begin{verbatim}
# As zimbra
zmprov modifyAccount account@example.tld {propertyName} {value}
\end{verbatim}

\begin{description}
\item [zimbraChatServiceEnabled] \verb+[boolean]+, Default value: \verb+true+.

Enable the Chat Service.

Can be applied to:
\begin{itemize}
    \item Global
    \item Server
\end{itemize}

\item [zimbraChatHistoryEnabled] \verb+[boolean]+, Default value: \verb+true+, requires a mailbox restart to be applied.

Enable the chat history writing inside the chat folder.

Can be applied to:
\begin{itemize}
    \item Cos
    \item Account
\end{itemize}

\item [zimbraChatConversationAuditEnabled] \verb+[boolean]+, Default value: \verb+false+.

Enable the dedicated log for the chat conversations.

Can be applied to:
\begin{itemize}
    \item Global
    \item Domain
\end{itemize}

\item [zimbraChatXmppSslPortEnabled] \verb+[boolean]+, Default value: \verb+false+, requires a mailbox restart to be applied.

Enable the XMPP legacy SSL port.

Can be applied to:
\begin{itemize}
    \item Global
    \item Server
\end{itemize}

\item [zimbraChatAllowUnencryptedPassword] \verb+[boolean]+, Default value: \verb+false+.

Allow unencrypted password login via XMPP.

Can be applied to:
\begin{itemize}
    \item Global
    \item Server
\end{itemize}

\item [zimbraChatXmppPort] \verb+[port]+, Default value: \verb+5222+, requires a mailbox restart to be applied.

The XMPP standard port, usually used with StartTLS.

Can be applied to:
\begin{itemize}
    \item Global
    \item Server
\end{itemize}

\item [zimbraChatXmppSslPort] \verb+[port]+, Default value: \verb+5223+, requires a mailbox restart to be applied.

The XMPP legacy SSL port.

Can be applied to:
\begin{itemize}
    \item Global
    \item Server
\end{itemize}

\item [zimbraChatAllowDlMemberAddAsFriend] \verb+[boolean]+, optional.

Add every member of the distribution list as buddies to eachother.

Can be applied to:
\begin{itemize}
    \item Distribution list
\end{itemize}

\end{description}
