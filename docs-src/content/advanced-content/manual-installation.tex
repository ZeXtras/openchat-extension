\section{Manual Installation}

\begin{warning}
    Manual installation is not supported.

    The OpenChat Zimlet and Extension are installed during the the Zimbra installation.

    Any modification to the installed zimbra packages or to the database may lead to a fail during the Zimbra upgrade.
\end{warning}

This chapter will guide the administrator through the manual installation of Zimbra OpenChat.

Zimbra OpenChat is designed specifically for Zimbra 8.8.0+, otherwise should be considered at `beta'
stage and you should not install it in a production environment.

\subsection{Requirements}
    \begin{itemize}
        \item Zimbra 8.8.0+
        \item 25MB of free space on disk
        \item Administrative access on Zimbra
        \item A modern browser (like Firefox 35+ or Google Chrome 45+)
    \end{itemize}

\subsection{Preliminary setup}

    Get and build the sources as described in the \autoref{sect:buildfromsrcs}.

\subsubsection{Networking}

    Make sure these ports are opened and exposed for a fully functional OpenChat server:
    \begin{description}
        \item [5269] XMPP Server-to-Server communication. Used between servers on the same cluster.
        \item [5222] XMPP Default server port, see \autoref{sect:confkeys}
        \item [5223] XMPP Default legacy SSL port, see \autoref{sect:confkeys}
    \end{description}

    If the mailbox is into a multi-server installation, make sure the servers can see and connect to each other.

\subsubsection{Database}

    Zimbra OpenChat Extension uses the Zimbra database to store some informations like:
    \begin{itemize}
        \item Information of the users
        \item Relations between the users (the buddy list information)
        \item Event messages which will be delivered and/or stored in the chat history (if enabled, see \autoref{sect:confkeys})
    \end{itemize}

    To create the database and the tables required by the OpenChat Extension type these commands:
    \begin{verbatim}
# As zimbra
mysql <<\EOF
  CREATE DATABASE `chat`
    DEFAULT CHARACTER SET `utf8`
    DEFAULT COLLATE `utf8_general_ci`;
  GRANT ALL PRIVILEGES ON chat.* TO zimbra;
  USE chat;
  CREATE TABLE `USER` (
    `ID` int(11) NOT NULL AUTO_INCREMENT,
    `ADDRESS` varchar(256) NOT NULL,
    PRIMARY KEY (`ID`)
  );
  CREATE TABLE `RELATIONSHIP` (
    `USERID` int(11) NOT NULL,
    `TYPE` tinyint(4) NOT NULL,
    `BUDDYADDRESS` varchar(256) NOT NULL,
    `BUDDYNICKNAME` varchar(128) NOT NULL,
    `GROUP` varchar(256) NOT NULL DEFAULT ''
  );
  CREATE TABLE `EVENTMESSAGE` (
    `ID` int(11) NOT NULL AUTO_INCREMENT,
    `USERID` int(11) NOT NULL,
    `EVENTID` varchar(36) DEFAULT NULL,
    `SENDER` varchar(256) NOT NULL,
    `TIMESTAMP` bigint(20) DEFAULT NULL,
    `MESSAGE` text,
    PRIMARY KEY (`ID`)
  );
EOF
    \end{verbatim}

\subsection{Installation of the packages}
\label{sect:installpkgs}

    To install the Extension packages copy them into the Zimbra class path. The Zimlet package need to be installed like
    any other Zimlet for Zimbra.

    Copy the Zimbra Extension packages:
    \begin{verbatim}
# As root
mkdir -p /opt/zimbra/lib/ext/openchat
cp /tmp/OpenChat/build/extension/openchat.jar /opt/zimbra/lib/ext/openchat/
cp /tmp/OpenChat/build/extension/zal.jar /opt/zimbra/lib/ext/openchat/
    \end{verbatim}

    Once the packages are in the right place is required a mailbox restart to load the extension:
    \begin{verbatim}
# As zimbra
zmmailboxdctl restart
    \end{verbatim}

    Deploy the Zimlet package:
    \begin{verbatim}
# As zimbra
zmzimletctl deploy /tmp/OpenChat/build/zimlet/com_zextras_chat_open.zip
    \end{verbatim}

    By default the zimlet is enabled for the Zimbra `default' class of service (COS).
    Enable the zimlet on the required COSes from the administration console.
