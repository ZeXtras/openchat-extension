\chapter{Introduction}

\section{About this document}
    This document targets system administrators, aiming to provide both an in-depth view of the product architecture in
    order to simplify troubleshooting and the knowledge necessary for solid customizations.

\section{Overview}
    Zimbra OpenChat is included in Zimbra since version 8.7.6 as Beta software and as Stable starting from Zimbra 8.8.0,
    its goal is to allow a simple and direct text-only chat between Zimbra accounts.
    It is composed by a Zimlet and a Zimbra extension.

    The Zimbra extensione expose an XMPP client to allow users to use the chat system from a standard XMPP client instead
    of using the Zimlet through the Zimbra Web Client. Users can use both methods, also at the same time.

\subsection{General architecture}
    Zimbra OpenChat is designed for Zimbra.
    It runs as Zimbra Server Extension which implements an XMPP serverand handle all the events shared between the clients.
    The OpenChat Zimbra Server Extension uses both XMPP and a custom protocol to communicate with the OpenChat Zimlet.

    The server extension can handle multiple sessions for a single user, withouth limits.
    Each session has his own delivery queue in order to be sure that each session has exactly all of the events like any
    other session of the user.

    Sessions are not limited by type (XMPP or OpenChat Zimlet).
    A user can use both the Zimbra Web Client and an XMPP client at the same time, without worrying about interferences
    between the sessions.

\subsection{Components}
    Zimbra OpenChat is compose by two parts:
    \begin{itemize}
        \item OpenChat Zimlet
        \item OpenChat Zimbra Extension
    \end{itemize}

    As the OpenChat Zimbra Extension does not require the OpenChat Zimlet to work properly, the OpenChat Zimlet is designed
    and built to work only with the OpenChat Zimbra Extension.

\subsubsection{Zimlet}
    The OpenChat Zimlet is the client component of the OpenChat. It run on the Zimbra Web Client and is managed like any
    other Zimlet.
    It is fully integrated in the Zimbra Web Client and uses the same graphics libraries.
    The Zimlet can work only with the OpenChat Extension, Zimlet and Extension uses a specific protocol to send and receive
    events.
    The browser compatibility is supported for the browsers and versions supported by the Zimbra Web Client.

    Feautres:
    \begin{itemize}
        \item Zimbra Web Client integration.
        \item Manage the buddy list. Users can add, edit and remove buddies from the buddy list.
        \item Visualize each `chat room' in a separate panel.
        \item Manage the personal status.
        \item Send/Receive plain messages from the users defined in the buddy list.
        \item Send the presence status to the users in the buddy list.
        \item Visualize the presence status of the users defined in the buddy list.
        \item Support for Emojis (emoticons) on conversations. Emojis are provided free by EmojiOne \url{http://emojione.com/}.
        \item Users can send an entire conversation as email.
        \item Search conversation in the chat history (if enabled in the Extension options, see \autoref{sect:confkeys}).
        \item Get desktop notifications for any incoming message.
    \end{itemize}

\subsubsection{Extension}
    The OpenChat Zimbra Extension is the core of OpenChat. It is a complete chat server which manage the events between
    the connected clients.

    It uses the ZAL\footnotemark[1] to integrate into Zimbra.

    The extension can handle two types of connections:
    \begin{itemize}
        \item SOAP connections incoming from the OpenChat Zimlet, using Zimbra's SOAP infrastructure.
        \item XMPP connections from compatible clients.
    \end{itemize}

    Features:
    \begin{itemize}
        \item Handle events from multiple sessions.
        \item Handle SOAP connections for OpenChat Zimlet clients.
        \item Handle XMPP (plain and SSL) connections for compatible clients throught a public port (see \autoref{sect:confkeys})
        \item Store the chat history for each conversation into a dedicated `chat' folder of the user mailbox (if enabled, see \autoref{sect:confkeys}).
        \item Store the relationships between users into the Zimbra Database.
    \end{itemize}

    \footnotetext[1]{
        ZAL is the library that provides an Abstraction Layer for Zimbra Collaboration Server, allowing anyone to write
        highly maintainable Zimbra extensions that will run on any Zimbra server, regardless of the version. For more
        informations please visit \url{https://openzal.org/}
    }
