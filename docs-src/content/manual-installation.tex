\chapter{Manual Installation}

This chapter will guide the administrator through the installation of the OpenChat.

OpenChat is designed specifically for Zimbra 8.8.0+, otherwise should be considered at 'beta' stage and you should not install it in a production environment.

OpenChat requires a new database where are stored:
\begin{itemize}
\item The event messages
\item The relationships between the users
\item The users information
\end{itemize}

\section{Requirements}

\begin{itemize}
\item Zimbra 8.8.0+
\item 25MB of free space on disk
\item Administrative access on Zimbra
\end{itemize}

\section{Prepare the Database}
To create the database and the tables required by the OpenChat Extension:
\begin{verbatim}
mysql <<\EOF
  CREATE DATABASE `chat`
    DEFAULT CHARACTER SET `utf8`
    DEFAULT COLLATE `utf8_general_ci`;
  GRANT ALL PRIVILEGES ON chat.* TO zimbra;
  USE chat;
  CREATE TABLE `USER` (
    `ID` int(11) NOT NULL AUTO_INCREMENT,
    `ADDRESS` varchar(256) NOT NULL,
    PRIMARY KEY (`ID`)
  );
  CREATE TABLE `RELATIONSHIP` (
    `USERID` int(11) NOT NULL,
    `TYPE` tinyint(4) NOT NULL,
    `BUDDYADDRESS` varchar(256) NOT NULL,
    `BUDDYNICKNAME` varchar(128) NOT NULL,
    `GROUP` varchar(256) NOT NULL DEFAULT ''
  );
  CREATE TABLE `EVENTMESSAGE` (
    `ID` int(11) NOT NULL AUTO_INCREMENT,
    `USERID` int(11) NOT NULL,
    `EVENTID` varchar(36) DEFAULT NULL,
    `SENDER` varchar(256) NOT NULL,
    `TIMESTAMP` bigint(20) DEFAULT NULL,
    `MESSAGE` text,
    PRIMARY KEY (`ID`)
  );
EOF
\end{verbatim}

\section{Install the Zimbra Extension}
\begin{verbatim}
# As root
mkdir -p /opt/zimbra/lib/ext/openchat
cp /tmp/openchat/openchat.jar /opt/zimbra/lib/ext/openchat/
cp /tmp/openchat/zal.jar /opt/zimbra/lib/ext/openchat/
\end{verbatim}
Once the packages are in the right place is required a mailbox restart to load the extension:
\begin{verbatim}
# As zimbra
zmmailboxdctl restart
\end{verbatim}

\section{Install the Zimlet}
\begin{verbatim}
# As zimbra
zmzimletctl deploy /tmp/openchat/com_zextras_chat_open.zip
\end{verbatim}

\section{Enable the Zimlet}
By default the zimlet is enabled for the `default` COS.
Enable the zimlet on the required COSes from the administration console.