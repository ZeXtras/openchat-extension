\section{Zimlet}
The zimlet is the client component which the final user is using.\\
Its source code is available under GPLv2 license on github: \url{http://github.com/ZeXtras/openchat-zimlet}

\ifdefined\INTERNALDOC % Rendered only when an internal documentation is built
\subsection{Architecture}
The OpenChat Zimlet was designed to be semantically compatible with XMPP but it uses SOAP requests to communicate with
 the OpenChat Extension.

The chat client will register a new session, and start the ping requests.

Ping requests are pending unanswered HTTP requests, once the server receive a new update valuable for the client (eg.
 a message from another user) it will be sent to the client by answering the ping. As soon as the client receive
 the ping answer it will process the update (eg. the client can open a new chat window with the message).

Once the client has received ant event, a new ping request is done immediately.

If there is no information for the client within  about 30 seconds the answer will be empty. If no ping requests are issued
 within 1 minute, the server will expire the session and the user is marked as 'offline'.

If the same client try to reconnect with the same token the server will send to the client an event to tell about the session
 expiration. Receiving that event the client will create another session.

Each Ping answer also contain Zimbra own header (the same used by NoOp, and any other SOAP request), meaning that it'll
 be used by zimbra also as a notification vector for another means like new emails and calendars updates.
 A side effect of this behavior is the reduced NoOp requests made by the Zimbra IO.

If multiple computers (or browser tabs) are opened on the same zimbra account, the chat client will create an indipendent
 session for each, every session will be kept up to date with the same events. In case of multiple session
 the client shows the 'most available' status of every session.

Any action taken by the client will issue its own SOAP request, such as status change and new messages.
\fi

\section{Extension}
The zimbra extension is the server component, it handle the SOAP service and the XMPP services.\\
Its source code is available under GPLv2 license on github: \url{http://github.com/ZeXtras/openchat-extension}

\ifdefined\INTERNALDOC % Rendered only when an internal documentation is built
\subsection{XMPP Server}
Zimbra OpenChat does implement a simple and standard XMPP server which can be used by any standard XMPP client.

The server only implements few of the hundreds possible XEP (Xmpp Extension protocol, \url{http://xmpp.org/extensions/index.html} for XMPP,
 most relevant is carbon copy, which is useful for copy your own messages from web client to mobile xmpp application.

The server starts when your mailboxd starts and will listen to port \verb+5222+, which is the original unencrypted port
 which support StartTLS and the \verb+5223+ port, which is the legacy SSL only port.

These ports can be configured using zimbraChatXmppSslPortEnabled. % TODO: Improve this part

The authentication is against zimbra account password, and does by default require encryption to be enabled before authenticating.

When multiple mailboxd are involved events (or stanzas, in XMPP language) are sent through the port \verb+5269+,
 which is the official XMPP port for server to server communication.
 This port is not configurable and it only allow messages coming from the same zimbra infrastructure, XMPP federation is currently unsupported.
\fi
