\section{F.A.Q.}
\label{sect:faq}

    \subsection[Zimlet Issues]{OpenChat Zimlet Issues}

    \subsubsection{The OpenChatZimlet is not working after the user login and I see some JavaScript Errors, what can I do?}
        Most commonly is caused by caching issues, refresh all the caches with these commands:
        \begin{verbatim}
# As zimbra
zmprov flushCache -a zimlet com_zextras_chat_open
        \end{verbatim}
        If the problem persists, escalate the issue.

    \subsubsection{OpenChat Zimlet doesn’t start at login and a popup appears informing the user that the server is not
    available, what can I do?}
        \begin{info}
            Remeber that the OpenChat Zimlet will not start if the logged user is logged into using the delegated access
            feature (eg. View Mail button from the admin console) to protect the privacy of the user.
        \end{info}

        Check if the openchat extension is loaded correctly in the \verb+mailbox.log+ (see in \autoref{sect:mailboxlog}
        how to read the \verb+mailbox.log+).

        The loading of the Zimbra Extension is granted by the following lines at the mailbbox startup:
        \begin{verbatim}
xxxx-xx-xx xx:xx:xx,xxx INFO  [main] [] mailbox - OpenChat starting ...
xxxx-xx-xx xx:xx:xx,xxx INFO  [main] [] extensions - OpenChat started
        \end{verbatim}
        Otherwise an exception, escalate the issue sending the exception to the support team.

    \subsubsection{Another Zimlet is using the sidebar and a user can not see the OpenChat's buddy list, what can I do?}
        OpenChat Zimlet uses a container which can be used by other Zimlets. To void collisions the OpenChat Zimlet try
        to detect if that container is used or not.

        OpenChat Zimlet use an internal `black list' to detect the incompatible zimlets and disable the sidebar mode, switching
        to the docked mode.

        The detection may fails if the Zimlet which are using the sidebar container is not indexed into the internal
        blacklist.

        Please escalate the issue to the support team sending us the package name of the zimlet.

        If a user is stuck in the sidebar mode and anoter Zimlet has took the control of the siedebar You can reset the
        Zimlet user setting to use the docked mode with these commands:
        \begin{verbatim}
# As zimbra
# Reset the involved zimlet user preference:
zmprov modifyAccount user@example.com \
    -zimbraZimletUserProperties "com_zextras_chat_open:zxchat_pref_dockmode:FALSE"
zmprov modifyAccount user@example.com \
    -zimbraZimletUserProperties "com_zextras_chat_open:zxchat_pref_dockmode:TRUE"
# Set the zimlet user preference to dock mode:
zmprov modifyAccount user@example.com \
    +zimbraZimletUserProperties "com_zextras_chat_open:zxchat_pref_dockmode:TRUE"
        \end{verbatim}

        Then reload the Zimbra Web Client to apply the sttings modifications.

        If the problem persists, escalate the issue.

    \subsection[Extension Issues]{OpenChat Extension Issues}

    \subsubsection{Server to server messages are not delivered between two server, what can I do?}
        This issue can be caused by connections issues between two mailboxes. Verify if the port \verb+5269+ is opened on
        each server and the servers can connect to each other.

        In order to verify if the port is opened on the server, a simple check can be done trying to connecto to the \verb+5269+
        port using a telnet client.

        If everything seems to work properly, open the \verb+mailbox.log+ (see in \autoref{sect:mailboxlog} how to read
        the \verb+mailbox.log+) on both server and try to send an event (eg. a text message should be enough). If an
        exception appear take a look at that to give an hint on the error. If no meaningful exception, escalate the issue
        sending the exception to the support team.

    \subsection[Escalate an issue]{How to escalate an issue}
    \label{sect:how_to_escalate_an_issue}
        In case You have found an issue and You are not able to fix it, contact the support team with these informations:
        \begin{itemize}
            \item A detailed description of the issue: What you are expecting and what is really happening.
            \item A detailed description of the steps to reproduce the issue.
            \item A detailed description of the installation and the environment: (see \autoref{sect:gatheringinfo})
            \begin{itemize}
                \item Server information: Cpu, Ram, Number of the servers and for each server:
                \begin{itemize}
                    \item Zimbra Version (see \autoref{sect:gatheringinfo-zversion})
                    \item OpenChat Version
                    \item List of the installed zimlets (see \autoref{sect:gatheringinfo-listzimlets})
                \end{itemize}
                \item Client information:
                \begin{itemize}
                    \item Browser name and version
                    \item Connectivity used between the servers and the client
                    \item Client Skin (theme)
                    \item Client Language
                    \item List of the Zimlets enabled on the user (see \autoref{sect:gatheringinfo-userzimlets})
                \end{itemize}
            \end{itemize}
            \item Any log envolved by the issue:
            \begin{itemize}
                \item \verb+mailbox.log+ (see \autoref{sect:mailboxlog})
            \end{itemize}
            You can remove any personal information to protect Your and Your Users privacy.
        \end{itemize}
